\newpage
\begin{center}
    \begin{singlespace}
   	  \large \titleEn \\[0.8cm]
    \end{singlespace}

    % \begin{singlespace}
    % \begin{tabular}{>{\raggedright} l >{\raggedleft} r}
    Student : \studentEn  
    \hfill
    Advisor : Dr. \advisorEn \\[0.5cm]
    % \end{tabular}
    % \end{singlespace}

   \begin{singlespace}
   \large \instituteEn{} of\\ \universityEn \\[0.5cm]
   \end{singlespace}
    
   \LARGE ABSTRACT \\[0.5cm]	
\end{center}

\normalsize 
A controller based on geometric tracking controller has been developed for controlling a multirotor unmanned aerial vehicle with defective thrust system. The defective thrust can be caused by many reasons, such as defective motors and deformed propellers. To improve the flight performance of the mutirotor with defective thrust system, knowing efficiency of thrust on each motor is necessary in this research and it is estimated by UKF. With estimated efficiency of thrust on each motor, the designed controller can get the estimated efficiency of thrust as feedback from the UKF estimator and improve the flight performance of the mutirotor with defective thrust system. Noted that defective thrust and the feedback of estimated efficiency of thrust are all treated as an uncertainty and will cause uniformly ultimately bounded(UUB) in the controller during flight. A stability analysis was conducted to ensure that with an uncertainty the tracking errors would not diverge and the observability of UKF model was proven. By designing the UKF estimator and high gain geometric tracking controller, the flight performance has been verified in simulations and experiments. \\[0.7cm]

Keywords: multirotor unmanned aerial vehicle, fault-tolerant control of the multirotor, geometric tracking controller, parameters estimation, unscented kalman filter